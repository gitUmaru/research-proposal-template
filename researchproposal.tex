\documentclass{IEEEtran}

\usepackage{enumitem}
\usepackage[]{biblatex}
\usepackage[english]{babel}
\usepackage{graphicx}
\usepackage{indentfirst}
\usepackage{csquotes}
\graphicspath{ {./images/} }
\usepackage{amsmath,amssymb,amsfonts}
\usepackage{algorithmic}
\usepackage{textcomp}
\usepackage{xcolor}

\bibliography{bibliography}

\usepackage{amsmath}

\title{\LARGE Using a Deep Convolutional Generative Adversarial Network to Synthesize Type IV-V Skin Lesion Images
    \\
    \large Establishing the Research Project}

\author{
\IEEEauthorblockN{Muhammad Umar Ali} \\
\IEEEauthorblockA{\textit{School of Biomedical Engineering} \\
\textit{University of British Columbia} \\
umaruali@student.ubc.ca}
}

\begin{document}
    \maketitle
    \begin{abstract}
        In A systematic review of pictures in scientific articles describing skin manifestations associated with COVID-19 and other dermatological issues, 93\% (120 out of 130) were taken with patients with the three fairest skin tones (Types I-III) and no representation of darker skin tones (Type V and VI) \cite{lester}.
    \end{abstract}

    \section{Overview}
    { In this section, the essence of the proposed work is described by answering four key questions: what is objective of this study, why does the objective need to be addressed, what is the proposed solution, and why the solution is sufficient/expected results.}

    \subsection{What is the problem being addressed?}
    {In current news media there has been a lot of discussion about the dermatological symptoms associated with COVID-19 infection, even if the individual is otherwise asymptomatic. }
    \begin{figure}[htbp]
        \centering
        \includegraphics[width=8cm]{figure1}
        \caption{Atopic dermatitis in an infant with Type V-VI dark skin (\textbf{A}) compared to one with Type I-III fair skin (\textbf{B}) \cite{bolognia}}
    \end{figure}


\section{Deconstruction of The Solution}
{In this section, a brief plan of the proposed solution is drafted. A high level overview of the data pipeline is provided for a better sense of how the project will proceed.}
         \subsection{DCGAN}
             \begin{enumerate}
                 \item The Generator
                    \begin{enumerate}
                        \item {The generator uses upsampling layers to produce an image from a seed (random noise). The model uses LeakyReLU activation for each layer, except the output layer which uses tanh.}
                    \end{enumerate}
             \end{enumerate}

\section{Relevant Related Work}
    {In this section, identified related work is described. A general understanding of the current work being developed can help determine areas of further exploration.}
    \subsection{Skin Lesion Images of People of Colour}
        {In order to fully understand how to synthesize images of skin lesions and skin disease for people of colour, one must also understand how these diseases are presented in Type I-III skin tones compared to Type IV-VI.}

  \newpage
  \printbibliography
\end{document}
